%%
%% This is file `sample-sigplan.tex',
%% generated with the docstrip utility.
%%
%% The original source files were:
%%
%% samples.dtx  (with options: `all,proceedings,bibtex,sigplan')
%%
%% IMPORTANT NOTICE:
%%
%% For the copyright see the source file.
%%
%% Any modified versions of this file must be renamed
%% with new filenames distinct from sample-sigplan.tex.
%%
%% For distribution of the original source see the terms
%% for copying and modification in the file samples.dtx.
%%
%% This generated file may be distributed as long as the
%% original source files, as listed above, are part of the
%% same distribution. (The sources need not necessarily be
%% in the same archive or directory.)
%%
%%
%% Commands for TeXCount
%TC:macro \cite [option:text,text]
%TC:macro \citep [option:text,text]
%TC:macro \citet [option:text,text]
%TC:envir table 0 1
%TC:envir table* 0 1
%TC:envir tabular [ignore] word
%TC:envir displaymath 0 word
%TC:envir math 0 word
%TC:envir comment 0 0
%%
%% The first command in your LaTeX source must be the \documentclass
%% command.
%%
%% For submission and review of your manuscript please change the
%% command to \documentclass[manuscript, screen, review]{acmart}.
%%
%% When submitting camera ready or to TAPS, please change the command
%% to \documentclass[sigconf]{acmart} or whichever template is required
%% for your publication.
%%
%%
\documentclass[sigplan,screen]{acmart}
%%
%% \BibTeX command to typeset BibTeX logo in the docs
\AtBeginDocument{%
  \providecommand\BibTeX{{%
    Bib\TeX}}}

%% Rights management information.  This information is sent to you
%% when you complete the rights form.  These commands have SAMPLE
%% values in them; it is your responsibility as an author to replace
%% the commands and values with those provided to you when you
%% complete the rights form.
\setcopyright{acmlicensed}
\copyrightyear{2025}
\acmYear{2025}
\acmConference[Data Viz 25]{Data Visualization Course}{2025}{Luxembourg}
\acmDOI{}
\acmISBN{}

%%  Uncomment \acmBooktitle if the title of the proceedings is different
%%  from ``Proceedings of ...''!
%%
%%\acmBooktitle{Woodstock '18: ACM Symposium on Neural Gaze Detection,
%%  June 03--05, 2018, Woodstock, NY}


%%
%% Submission ID.
%% Use this when submitting an article to a sponsored event. You'll
%% receive a unique submission ID from the organizers
%% of the event, and this ID should be used as the parameter to this command.
%%\acmSubmissionID{123-A56-BU3}

%%
%% For managing citations, it is recommended to use bibliography
%% files in BibTeX format.
%%
%% You can then either use BibTeX with the ACM-Reference-Format style,
%% or BibLaTeX with the acmnumeric or acmauthoryear sytles, that include
%% support for advanced citation of software artefact from the
%% biblatex-software package, also separately available on CTAN.
%%
%% Look at the sample-*-biblatex.tex files for templates showcasing
%% the biblatex styles.
%%

%%
%% The majority of ACM publications use numbered citations and
%% references.  The command \citestyle{authoryear} switches to the
%% "author year" style.
%%
%% If you are preparing content for an event
%% sponsored by ACM SIGGRAPH, you must use the "author year" style of
%% citations and references.
%% Uncommenting
%% the next command will enable that style.
%%\citestyle{acmauthoryear}


%%
%% end of the preamble, start of the body of the document source.
\begin{document}

%%
%% The "title" command has an optional parameter,
%% allowing the author to define a "short title" to be used in page headers.
\title{Interactive Multi-Dimensional Housing Data Visualization with Coordinated Views}

%%
%% The "author" command and its associated commands are used to define
%% the authors and their affiliations.
%% Of note is the shared affiliation of the first two authors, and the
%% "authornote" and "authornotemark" commands
%% used to denote shared contribution to the research.

\author{Jan Esquivel Marxen}
\affiliation{%
  \institution{University of Luxembourg}
    \country{Luxembourg}
}
\email{jan.esquivel.001@student.uni.lu}

%%
%% By default, the full list of authors will be used in the page
%% headers. Often, this list is too long, and will overlap
%% other information printed in the page headers. This command allows
%% the author to define a more concise list
%% of authors' names for this purpose.
\renewcommand{\shortauthors}{Marxen}

%%
%% The abstract is a short summary of the work to be presented in the
%% article.
\begin{abstract}
  Effective visualization of multi-dimensional datasets requires
  coordinated interaction patterns that enable users to explore
  relationships between continuous and categorical variables.
  This report presents an interactive visualization system for
  housing market data implementing synchronized scatterplot and
  violin plot views with multi-rectangle brushing. The left view
  displays area-price relationships through a traditional
  scatterplot, while the right view presents price distributions
  across categorical counts (bedrooms, bathrooms, stories) using
  kernel density estimation. The system demonstrates React-D3.js
  integration patterns with parameterized violin rendering
  supporting n variables, dynamic color palettes, and persistent
  multi-rectangle selection enabling non-linear subset queries.
  We justify the visual encodings based on data properties and
  user tasks, discuss architectural decisions including the React
  container pattern and shared state management, and evaluate the
  trade-offs of design choices including KDE bandwidth selection
  and multi-brush interaction complexity.
\smallskip
\noindent\rule{\linewidth}{0.4pt}
\textit{\textbf{Note:}} Portions of this report were developed with
the assistance of artificial intelligence tools to support code
generation, documentation, and readability improvements. All technical
content and implementations were reviewed and validated by the author.

\end{abstract}

%%
%% The code below is generated by the tool at http://dl.acm.org/ccs.cfm.
%% Please copy and paste the code instead of the example below.
%%
\begin{CCSXML}
<ccs2012>
   <concept>
       <concept_id>10003120.10003121.10003122.10003334</concept_id>
       <concept_desc>Human-centered computing~Interactive systems and 
       tools</concept_desc>
       <concept_significance>500</concept_significance>
   </concept>
   <concept>
       <concept_id>10003120.10003121.10003124</concept_id>
       <concept_desc>Human-centered computing~Visualization</concept_desc>
       <concept_significance>500</concept_significance>
   </concept>
   <concept>
       <concept_id>10003120.10003121.10003124.10003125</concept_id>
       <concept_desc>Human-centered computing~Visualization 
       techniques</concept_desc>
       <concept_significance>300</concept_significance>
   </concept>
 </ccs2012>
\end{CCSXML}
%%
%% Keywords. The author(s) should pick words that accurately describe
%% the work being presented. Separate the keywords with commas.
\keywords{Data Visualization, D3.js, React, Information Visualization,
          Kernel Density Estimation, Violin Plots, Coordinated Views,
          Brushing and Linking}

%%
%% This command processes the author and affiliation and title
%% information and builds the first part of the formatted document.
\maketitle

\section{Introduction}

Exploratory analysis of multi-dimensional datasets benefits from
visualization techniques that reveal patterns across both continuous
and categorical variables while supporting interactive investigation.
Housing market data presents a rich domain for visualization: prices
depend on continuous variables (property area) and discrete structural
characteristics (bedrooms, bathrooms, stories). Traditional scatterplots
show bivariate relationships effectively but provide limited insight
into distributional patterns across categories.

This project implements coordinated multiple views combining a
scatterplot with violin plot distributions:

\textbf{Left View - Scatterplot with 2D Brush:}
\begin{itemize}
\item Area vs. Price with multi-rectangle persistent brushing
\item Manual rectangle creation supporting complex queries
\item Click selection for individual data point inspection
\end{itemize}

\textbf{Right View - Violin Plots with Interaction:}
\begin{itemize}
\item Price distributions for bedrooms, stories, bathrooms (1-5)
\item Kernel density estimation with Epanechnikov kernel
\item Jittered points overlay within violin contours
\item Multi-rectangle brushing synchronized with left scatterplot
\item Parameterized design supporting n categorical variables
\end{itemize}

Both views implement synchronized highlighting: selections in either
view highlight respective points in both. The architecture follows React-D3 best
practices with separate container components managing state and D3
classes handling rendering.

This report justifies visual encodings based on data properties and
exploratory tasks (Section~\ref{sec:design}), describes the React-D3
architecture and implementation patterns
(Section~\ref{sec:implementation}), and discusses design trade-offs
including KDE parameters, color encoding, and multi-brush complexity
(Section~\ref{sec:discussion}).

\section{Visual Design and Justification}
\label{sec:design}

\subsection{Dataset Properties}

The Housing.csv dataset contains 545 records describing residential
properties with the following attributes:

\begin{itemize}
\item \textbf{Continuous:} price (dependent variable), area (square
      feet)
\item \textbf{Discrete ordinal:} bedrooms, bathrooms, stories (counts
      1-5)
\item \textbf{Categorical:} furnishing status (furnished,
      semi-furnished, unfurnished)
\end{itemize}

\textbf{User Tasks:} The visualization supports exploratory analysis
tasks: (1) identifying area-price correlation patterns, (2)
comparing price distributions across bedroom/bathroom/story counts, (3)
investigating outliers and their categorical characteristics, (4)
selecting subsets matching non-linear criteria (e.g., ``3-bedroom houses
priced 2M-4M'').

\subsection{Scatterplot Design (Left View)}

\textbf{Encoding Rationale:}
\begin{itemize}
\item \textbf{X-axis (area):} Linear scale for continuous quantitative
      variable; position is the most effective channel for quantitative
      data
\item \textbf{Y-axis (price):} Linear scale shared between left and
      right views to enable direct comparison
\end{itemize}

\textbf{Interaction Design:}
\begin{itemize}
\item \textbf{Multi-rectangle brushing:} Users draw persistent
      rectangles via mouse movements and clicks. Multiple rectangles
      accumulate, enabling union queries (e.g., select both
      low-area/low-price AND high-area/high-price segments)
\item \textbf{Click selection:} Single-click on points for individual
      inspection
\item \textbf{Clear selection button:} Removes all brushes and resets selection
      state
\end{itemize}

\TODO: expand pros and cons
\\
\textbf{Design Pros:} Familiar scatterplot idiom; multi-rectangle brushing
supports more complex queries than only single-rectangle selection; 
position encoding maximizes precision.

\textbf{Design Cons:} Overplotting possible at high densities; no categorical
variable encoding (e.g., furnishing status not shown); multi-brush UI
requires learning.

\subsection{Violin Plot Design (Right View)}

\textbf{Encoding Rationale:}
\begin{itemize}
\item \textbf{X-axis:} Categorical band scale for bedroom/bathroom/
      story counts (1-5). Each x-position represents a count value; n
      violins are positioned evenly within each band.
\item \textbf{Y-axis:} Price (shared scale with left view, no
      redundant axis drawn)
\item \textbf{Violin contour:} Area encodes kernel density estimates
      of price distributions per category. Width at a given y-value
      represents relative frequency of that price.
\item \textbf{Points (jittered):} Individual data points overlaid
      within violin width to provide access to raw data.
\item \textbf{Color:} D3 Tableau10 palette assigns distinct hues to
      bedrooms/bathrooms/stories.
\item \textbf{Separators:} Dashed vertical lines between x-categories
      improve visual grouping.
\end{itemize}

\textbf{Interaction Design:}
Same pattern as left view; rectangles select points within violin plots.
Selections update both views simultaneously.
\end{itemize}

\textbf{Pros:} Violin plots show full distributions (not just summary
statistics like box plots); KDE reveals multi-modal patterns;
color-coded variables enable side-by-side comparison; points provide
individual data access; parameterized design supports
adding/removing variables dynamically.

\textbf{Cons:} Violin plots less familiar than box plots; KDE bandwidth
parameter affects smoothness (requires tuning); horizontal scrolling
needed for many variables; multi-modal distributions may be harder to
interpret than histograms for some users.

\subsection{Design Alternatives Considered}

\textbf{Parallel Coordinates:} Would show all variables simultaneously
but sacrifice distributional detail and suffer from overplotting at 545
records. Would also require scrolling for many variables. Is already well-known, 
whereas our goal was to show alternative visualizations.

\textbf{Heatmap/Matrix:} Could show bivariate distributions but would
not preserve individual data points and requires binning choices (not the 
point of this assignment).

\textbf{Small Multiples of Histograms:} Similar to violins but less
compact; histograms require bin width tuning and lose smooth
distributional curves. We would also not be able to overlay individual points easily.

The chosen design balances distributional insight (violins), individual
data access (points), and effective multivariate exploration (scatterplot)
while supporting non-linear interactive queries (multi-brush).

\section{Implementation and Architecture}
\label{sec:implementation}

\subsection{React-D3 Integration Pattern}

The system follows a clean separation between React (declarative state
management) and D3 (imperative SVG rendering):

\begin{verbatim}
App.js (React root)
  +-- ScatterplotContainer (React)
  |     +-- Scatterplot-d3.js (D3 class)
  +-- ViolinScatterContainer (React)
        +-- ViolinScatter-d3.js (D3 class)
\end{verbatim}

\textbf{App.js responsibilities:}
\begin{itemize}
\item CSV loading via \texttt{fetchCSV} utility
\item Compute shared \texttt{yDomain} for synchronized price scales
\item Manage \texttt{selectedItems} state (array of selected data
      objects)
\item Provide \texttt{scatterplotControllerMethods} callbacks
      (\texttt{updateSelectedItems}, \texttt{handleOnBrush})
\item Pass \texttt{violinVariables} array to right view
\end{itemize}

\textbf{Container components:}
\begin{itemize}
\item Create D3 class instance on mount (\texttt{useEffect})
\item Forward data, controller methods, and configuration to D3
\item Compute color palettes from variable arrays (\texttt{useMemo})
\item Render legends and UI controls (Clear button)
\item Call \texttt{highlightSelected} when \texttt{selectedItems}
      changes
\end{itemize}

\textbf{D3 classes:}
\begin{itemize}
\item \texttt{create(config)}: Build SVG structure, groups, pointer
      event handlers
\item \texttt{render(data, controllerMethods)}: Compute scales, KDE,
      draw marks
\item \texttt{highlightSelected(selectedItems)}: Update opacity/stroke
      for selected items
\item \texttt{clear()}: Cleanup on unmount
\end{itemize}

This pattern avoids common pitfalls: D3 instances are created once (not
on every render), callbacks prevent infinite re-render loops,
and imperative D3 logic stays isolated from React's declarative model.

\subsection{Multi-Rectangle Brushing}

Unlike \texttt{d3.brush} (single brush area), we implement persistent
multi-rectangle creation:

\begin{verbatim}
svgElem.on('mousedown.multiRect', (event) => {
  const [mx,my] = d3.pointer(event, this.svg.node());
  this._creating = true;
  this._createStart = [mx,my];
  this._currentRect = this.svg.append('rect')
    .attr('class','multi-brush')
    .attr('x', mx).attr('y', my)
    .attr('width', 0).attr('height', 0);
});
\end{verbatim}

On \texttt{mouseup}, the rectangle's bounding box is stored in
\texttt{this.multiBrushes}, and \texttt{updateBrushesSelection()}
computes the union of points inside all rectangles. Selected items are
passed to React via \texttt{controllerMethods.handleOnBrush(unique)},
triggering \texttt{highlightSelected()} in both views.

\section{Discussion and Trade-offs}
\label{sec:discussion}

\subsection{Multi-Brush Interaction Complexity}

\textbf{Pros:} Enables more complex queries (union of multiple ranges);
persistent rectangles provide visual feedback of selection criteria.

\textbf{Cons:} Non-standard interaction pattern; users must learn
mousedown-drag-mouseup workflow; no per-rectangle delete UI (only
global Clear); overlapping rectangles could confuse users.

\textbf{Improvement:} Adding individual 'x' buttons to each rectangle
or implementing shift-click to toggle rectangles would improve
usability; being able to zoom/pan the scatterplot would also help.

\subsection{Color Encoding vs. Small Multiples}

\textbf{Choice:} Three variables (bedrooms/bathrooms/stories) are
color-coded and overlaid within each x-category.

\textbf{Pros:} Compact layout; direct side-by-side comparison; color
legend links variables to hues.

\textbf{Cons:} Overlapping violins may occlude each other; color
discrimination harder than spatial separation; limited to ~10 variables
(palette size), unless we repeat colors.

\textbf{Alternative:} Small multiples (separate panels per variable)
would eliminate occlusion but sacrifice compactness and would make
cross-variable comparison harder.

\section{Conclusion}

This project demonstrates effective patterns for interactive
multi-dimensional visualization with React and D3.js. The coordinated
scatterplot-violin view combination enables rich exploratory analysis
of housing data, revealing both bivariate relationships and
distributional patterns across categories. Multi-rectangle brushing
supports complex subset queries, and the parameterized architecture
allows dynamic variable selection.

\textbf{Key contributions:}
\begin{itemize}
\item Clean React-D3 integration avoiding re-render pitfalls
\item Parameterized violin plot architecture supporting n variables
\item Multi-rectangle brushing for flexible subset selection
\item KDE utilities (helper.js) for reusable density estimation
\item Comprehensive inline documentation for maintainability
\end{itemize}

The design balances distributional insight (KDE violins), individual
data access (jittered points), and effective bivariate exploration
(scatterplot) while following established React patterns (useState,
useRef, useEffect, useMemo) and D3 idioms (scales, generators, join).
Future work could improve KDE bandwidth selection, add per-rectangle
deletion, and optimize performance for larger datasets.

%%
%% The next two lines define the bibliography style to be used, and
%% the bibliography file.
\bibliographystyle{ACM-Reference-Format}
%% Uncomment the following line if you have a bibliography file
%% \bibliography{sample-base}

\end{document}
\endinput