%%
%% This is file `sample-sigplan.tex',
%% generated with the docstrip utility.
%%
%% The original source files were:
%%
%% samples.dtx  (with options: `all,proceedings,bibtex,sigplan')
%%
%% IMPORTANT NOTICE:
%%
%% For the copyright see the source file.
%%
%% Any modified versions of this file must be renamed
%% with new filenames distinct from sample-sigplan.tex.
%%
%% For distribution of the original source see the terms
%% for copying and modification in the file samples.dtx.
%%
%% This generated file may be distributed as long as the
%% original source files, as listed above, are part of the
%% same distribution. (The sources need not necessarily be
%% in the same archive or directory.)
%%
%%
%% Commands for TeXCount
%TC:macro \cite [option:text,text]
%TC:macro \citep [option:text,text]
%TC:macro \citet [option:text,text]
%TC:envir table 0 1
%TC:envir table* 0 1
%TC:envir tabular [ignore] word
%TC:envir displaymath 0 word
%TC:envir math 0 word
%TC:envir comment 0 0
%%
%% The first command in your LaTeX source must be the \documentclass
%% command.
%%
%% For submission and review of your manuscript please change the
%% command to \documentclass[manuscript, screen, review]{acmart}.
%%
%% When submitting camera ready or to TAPS, please change the command
%% to \documentclass[sigconf]{acmart} or whichever template is required
%% for your publication.
%%
%%
\documentclass[sigplan,screen]{acmart}
%%
%% \BibTeX command to typeset BibTeX logo in the docs
\AtBeginDocument{%
  \providecommand\BibTeX{{%
    Bib\TeX}}}

%% Rights management information.  This information is sent to you
%% when you complete the rights form.  These commands have SAMPLE
%% values in them; it is your responsibility as an author to replace
%% the commands and values with those provided to you when you
%% complete the rights form.
\setcopyright{acmlicensed}
\copyrightyear{2025}
\acmYear{2025}
\acmConference[Data Viz 25]{Data Visualization Course}{2025}{Luxembourg}
\acmDOI{}
\acmISBN{}

%%  Uncomment \acmBooktitle if the title of the proceedings is different
%%  from ``Proceedings of ...''!
%%
%%\acmBooktitle{Woodstock '18: ACM Symposium on Neural Gaze Detection,
%%  June 03--05, 2018, Woodstock, NY}


%%
%% Submission ID.
%% Use this when submitting an article to a sponsored event. You'll
%% receive a unique submission ID from the organizers
%% of the event, and this ID should be used as the parameter to this command.
%%\acmSubmissionID{123-A56-BU3}

%%
%% For managing citations, it is recommended to use bibliography
%% files in BibTeX format.
%%
%% You can then either use BibTeX with the ACM-Reference-Format style,
%% or BibLaTeX with the acmnumeric or acmauthoryear sytles, that include
%% support for advanced citation of software artefact from the
%% biblatex-software package, also separately available on CTAN.
%%
%% Look at the sample-*-biblatex.tex files for templates showcasing
%% the biblatex styles.
%%

%%
%% The majority of ACM publications use numbered citations and
%% references.  The command \citestyle{authoryear} switches to the
%% "author year" style.
%%
%% If you are preparing content for an event
%% sponsored by ACM SIGGRAPH, you must use the "author year" style of
%% citations and references.
%% Uncommenting
%% the next command will enable that style.
%%\citestyle{acmauthoryear}


%%
%% end of the preamble, start of the body of the document source.
\begin{document}

%%
%% The "title" command has an optional parameter,
%% allowing the author to define a "short title" to be used in page headers.
\title{Interactive Multi-Dimensional Housing Data Visualization with Coordinated Views}

%%
%% The "author" command and its associated commands are used to define
%% the authors and their affiliations.
%% Of note is the shared affiliation of the first two authors, and the
%% "authornote" and "authornotemark" commands
%% used to denote shared contribution to the research.

\author{Jan Esquivel Marxen}
\affiliation{%
  \institution{University of Luxembourg}
    \country{Luxembourg}
}
\email{jan.esquivel.001@student.uni.lu}

%%
%% By default, the full list of authors will be used in the page
%% headers. Often, this list is too long, and will overlap
%% other information printed in the page headers. This command allows
%% the author to define a more concise list
%% of authors' names for this purpose.
\renewcommand{\shortauthors}{Marxen}

%%
%% The abstract is a short summary of the work to be presented in the
%% article.
\begin{abstract}
  This report presents an interactive housing data visualization
  combining synchronized scatterplot and violin plot views with
  multi-rectangle brushing. The left scatterplot shows area-price
  relationships, while the right view displays price distributions
  across bedrooms, bathrooms, and stories using kernel density
  estimation. The system uses React-D3.js patterns with parameterized
  violin rendering that supports n variables and persistent
  multi-rectangle selection for complex, non-linear queries. We discuss the visual
  encoding choices, the React-D3 architecture, and trade-offs of this design.\\
\smallskip
\noindent\rule{\linewidth}{0.4pt}
\textit{\textbf{Note:}} Parts of this report were developed with AI
assistance for code generation and documentation. All technical content
was reviewed and validated by the author.

\end{abstract}


%%
%% This command processes the author and affiliation and title
%% information and builds the first part of the formatted document.
\maketitle

\section{Introduction}

Housing market data includes both continuous variables (area, price)
and discrete structural characteristics (bedrooms, bathrooms, stories).
While scatterplots effectively show bivariate relationships, they don't
reveal distributional patterns across categories.

This project implements coordinated views combining a scatterplot with
violin plots:

\textbf{Left View - Scatterplot with 2D Brush:}
\begin{itemize}
\item Area vs. Price with multi-rectangle persistent brushing
\item Manual rectangle creation supporting complex queries
\item Click selection for individual data point inspection
\end{itemize}

\textbf{Right View - Violin Plots with Interaction:}
\begin{itemize}
\item Price distributions for bedrooms, stories, bathrooms (1-5)
\item Kernel density estimation with Epanechnikov kernel
\item Jittered points overlay within violin contours
\item Multi-rectangle brushing synchronized with left scatterplot
\item Parameterized design supporting n categorical variables
\end{itemize}

Selections in either view highlight points in both views. The
architecture separates React containers (state management) from D3
classes (rendering).

Section~\ref{sec:design} justifies the visual encodings,
Section~\ref{sec:implementation} describes the implementation, and
Section~\ref{sec:discussion} discusses trade-offs.

\section{Visual Design and Justification}
\label{sec:design}

\subsection{Dataset Properties}

The Housing.csv dataset has 545 records with:

\begin{itemize}
\item \textbf{Continuous:} price, area (sq. ft.)
\item \textbf{Discrete ordinal:} bedrooms, bathrooms, stories (1-5)
\item \textbf{Categorical:} furnishing status, and others
\end{itemize}

\textbf{User Tasks:} Find area-price correlations, compare price
distributions across room counts, investigate outliers (which could benefit the user), and select
complex subsets (e.g., ``3-bedroom, 1 or 3 bathroom, 2 story houses priced 2M-4M'').

\subsection{Scatterplot Design (Left View)}

\textbf{Encoding:}
\begin{itemize}
\item \textbf{X-axis (area):} Linear scale; position encodes continuous
      data effectively
\item \textbf{Y-axis (price):} Linear scale shared with right view for
      comparison
\end{itemize}

\textbf{Interaction:}
\begin{itemize}
\item \textbf{Multi-rectangle brushing:} Draw persistent rectangles to
      select multiple regions (e.g., both low-price and high-price
      segments)
\item \textbf{Click selection:} Single-click to inspect individual
      points
\item \textbf{Clear button:} Remove all brushes
\end{itemize}

\textbf{Pros:} Familiar idiom; multi-rectangle selection enables complex
queries; precise position encoding.

\textbf{Cons:} Potential overplotting; no furnishing status encoding;
non-standard brush interaction.

\subsection{Violin Plot Design (Right View)}

\textbf{Encoding:}
\begin{itemize}
\item \textbf{X-axis:} Categorical bands for counts 1-5; n violins per
      band
\item \textbf{Y-axis:} Price (shared scale, axis hidden)
\item \textbf{Violin contour:} Width encodes KDE probability density
\item \textbf{Points (jittered):} Raw data overlaid within violins
\item \textbf{Color:} Tableau10 palette distinguishes variables
\item \textbf{Separators:} Dashed lines between categories
\end{itemize}

\textbf{Interaction:}
Same multi-rectangle brushing as scatterplot. Selections update both
views.

\textbf{Pros:} Shows full distributions with multimodal patterns;
color-coding enables comparison; jittered points show raw data;
parameterized for n variables.

\textbf{Cons:} Less familiar than box plots; KDE bandwidth needs tuning;
requires scrolling for many variables; harder to read than
histograms.

\subsection{Design Alternatives Considered}

\textbf{Parallel Coordinates:} Shows all variables but loses
distributional detail and suffers from overplotting. Also a standard
technique; we wanted something different.

\textbf{Heatmap:} Shows bivariate distributions but loses individual
points and requires binning.

\textbf{Histogram Small Multiples:} Less compact than violins; requires
bin tuning; can't easily overlay points.

Our design balances distributional insight, individual data access, and
interactive exploration.

\section{Implementation and Architecture}
\label{sec:implementation}

\subsection{React-D3 Integration Pattern}

We separate React (state management) from D3 (SVG rendering):

\begin{verbatim}
App.js (React root)
  +-- ScatterplotContainer (React)
  |     +-- Scatterplot-d3.js (D3 class)
  +-- ViolinScatterContainer (React)
        +-- ViolinScatter-d3.js (D3 class)
\end{verbatim}

\textbf{App.js:} Loads CSV, computes shared \texttt{yDomain}, manages
\texttt{selectedItems} state, provides callbacks.

\textbf{Containers:} Create D3 instances (\texttt{useEffect}), compute
color palettes (\texttt{useMemo}), render legends and controls, call
\texttt{highlightSelected} on selection changes.

\textbf{D3 classes:} \texttt{create()} builds SVG structure,
\texttt{render()} computes scales and draws, \texttt{highlightSelected()}
updates styling, \texttt{clear()} cleanup.

This avoids re-render loops by creating D3 instances once and keeping
imperative logic isolated.

\subsection{Multi-Rectangle Brushing}

Unlike \texttt{d3.brush} (single area), we support persistent
multi-rectangle selection:

\begin{verbatim}
svgElem.on('mousedown.multiRect', (event) => {
  const [mx,my] = d3.pointer(event, this.svg.node());
  this._creating = true;
  this._createStart = [mx,my];
  this._currentRect = this.svg.append('rect')
    .attr('class','multi-brush')
    .attr('x', mx).attr('y', my)
    .attr('width', 0).attr('height', 0);
});
\end{verbatim}

On \texttt{mouseup}, we store the rectangle and compute the union of
selected points. This triggers \texttt{highlightSelected()} in both
views.

\section{Discussion and Trade-offs}
\label{sec:discussion}

\subsection{Multi-Brush Interaction Complexity}

\textbf{Pros:} Enables complex queries; persistent rectangles show
selection criteria visually.

\textbf{Cons:} Non-standard interaction; no per-rectangle delete (only
global Clear); overlapping rectangles can be confusing.

\textbf{Improvements:} Add 'x' buttons per rectangle, shift-click to
toggle, and zoom/pan for the scatterplot.

\subsection{Color Encoding vs. Small Multiples}

We color-code three variables and overlay them within each x-category.

\textbf{Pros:} Compact; easy comparison; clear legend.

\textbf{Cons:} Overlapping violins occlude; color harder to distinguish
than position; limited palette (~10 colors).

\textbf{Alternative:} Small multiples avoid occlusion but use more space
and make comparison harder.

\section{Conclusion}

This project uses scatterplot and violin plot views with
multi-rectangle brushing to explore housing data in an interactive, non-linear way.
The coordinated views reveal bivariate relationships and distributional patterns across
categories.
The design balances distributional insight, raw data access, and
interactive exploration. Future improvements could include zooming,
per-rectangle deletion, and binary variable inclusion.

%%
%% The next two lines define the bibliography style to be used, and
%% the bibliography file.
\bibliographystyle{ACM-Reference-Format}
%% Uncomment the following line if you have a bibliography file
%% \bibliography{sample-base}

\end{document}
\endinput